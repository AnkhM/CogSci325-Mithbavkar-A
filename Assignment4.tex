% Options for packages loaded elsewhere
\PassOptionsToPackage{unicode}{hyperref}
\PassOptionsToPackage{hyphens}{url}
\PassOptionsToPackage{dvipsnames,svgnames,x11names}{xcolor}
%
\documentclass[
  letterpaper,
  DIV=11,
  numbers=noendperiod]{scrartcl}

\usepackage{amsmath,amssymb}
\usepackage{iftex}
\ifPDFTeX
  \usepackage[T1]{fontenc}
  \usepackage[utf8]{inputenc}
  \usepackage{textcomp} % provide euro and other symbols
\else % if luatex or xetex
  \usepackage{unicode-math}
  \defaultfontfeatures{Scale=MatchLowercase}
  \defaultfontfeatures[\rmfamily]{Ligatures=TeX,Scale=1}
\fi
\usepackage{lmodern}
\ifPDFTeX\else  
    % xetex/luatex font selection
\fi
% Use upquote if available, for straight quotes in verbatim environments
\IfFileExists{upquote.sty}{\usepackage{upquote}}{}
\IfFileExists{microtype.sty}{% use microtype if available
  \usepackage[]{microtype}
  \UseMicrotypeSet[protrusion]{basicmath} % disable protrusion for tt fonts
}{}
\makeatletter
\@ifundefined{KOMAClassName}{% if non-KOMA class
  \IfFileExists{parskip.sty}{%
    \usepackage{parskip}
  }{% else
    \setlength{\parindent}{0pt}
    \setlength{\parskip}{6pt plus 2pt minus 1pt}}
}{% if KOMA class
  \KOMAoptions{parskip=half}}
\makeatother
\usepackage{xcolor}
\setlength{\emergencystretch}{3em} % prevent overfull lines
\setcounter{secnumdepth}{-\maxdimen} % remove section numbering
% Make \paragraph and \subparagraph free-standing
\ifx\paragraph\undefined\else
  \let\oldparagraph\paragraph
  \renewcommand{\paragraph}[1]{\oldparagraph{#1}\mbox{}}
\fi
\ifx\subparagraph\undefined\else
  \let\oldsubparagraph\subparagraph
  \renewcommand{\subparagraph}[1]{\oldsubparagraph{#1}\mbox{}}
\fi


\providecommand{\tightlist}{%
  \setlength{\itemsep}{0pt}\setlength{\parskip}{0pt}}\usepackage{longtable,booktabs,array}
\usepackage{calc} % for calculating minipage widths
% Correct order of tables after \paragraph or \subparagraph
\usepackage{etoolbox}
\makeatletter
\patchcmd\longtable{\par}{\if@noskipsec\mbox{}\fi\par}{}{}
\makeatother
% Allow footnotes in longtable head/foot
\IfFileExists{footnotehyper.sty}{\usepackage{footnotehyper}}{\usepackage{footnote}}
\makesavenoteenv{longtable}
\usepackage{graphicx}
\makeatletter
\def\maxwidth{\ifdim\Gin@nat@width>\linewidth\linewidth\else\Gin@nat@width\fi}
\def\maxheight{\ifdim\Gin@nat@height>\textheight\textheight\else\Gin@nat@height\fi}
\makeatother
% Scale images if necessary, so that they will not overflow the page
% margins by default, and it is still possible to overwrite the defaults
% using explicit options in \includegraphics[width, height, ...]{}
\setkeys{Gin}{width=\maxwidth,height=\maxheight,keepaspectratio}
% Set default figure placement to htbp
\makeatletter
\def\fps@figure{htbp}
\makeatother

\KOMAoption{captions}{tableheading}
\makeatletter
\makeatother
\makeatletter
\makeatother
\makeatletter
\@ifpackageloaded{caption}{}{\usepackage{caption}}
\AtBeginDocument{%
\ifdefined\contentsname
  \renewcommand*\contentsname{Table of contents}
\else
  \newcommand\contentsname{Table of contents}
\fi
\ifdefined\listfigurename
  \renewcommand*\listfigurename{List of Figures}
\else
  \newcommand\listfigurename{List of Figures}
\fi
\ifdefined\listtablename
  \renewcommand*\listtablename{List of Tables}
\else
  \newcommand\listtablename{List of Tables}
\fi
\ifdefined\figurename
  \renewcommand*\figurename{Figure}
\else
  \newcommand\figurename{Figure}
\fi
\ifdefined\tablename
  \renewcommand*\tablename{Table}
\else
  \newcommand\tablename{Table}
\fi
}
\@ifpackageloaded{float}{}{\usepackage{float}}
\floatstyle{ruled}
\@ifundefined{c@chapter}{\newfloat{codelisting}{h}{lop}}{\newfloat{codelisting}{h}{lop}[chapter]}
\floatname{codelisting}{Listing}
\newcommand*\listoflistings{\listof{codelisting}{List of Listings}}
\makeatother
\makeatletter
\@ifpackageloaded{caption}{}{\usepackage{caption}}
\@ifpackageloaded{subcaption}{}{\usepackage{subcaption}}
\makeatother
\makeatletter
\@ifpackageloaded{tcolorbox}{}{\usepackage[skins,breakable]{tcolorbox}}
\makeatother
\makeatletter
\@ifundefined{shadecolor}{\definecolor{shadecolor}{rgb}{.97, .97, .97}}
\makeatother
\makeatletter
\makeatother
\makeatletter
\makeatother
\ifLuaTeX
  \usepackage{selnolig}  % disable illegal ligatures
\fi
\IfFileExists{bookmark.sty}{\usepackage{bookmark}}{\usepackage{hyperref}}
\IfFileExists{xurl.sty}{\usepackage{xurl}}{} % add URL line breaks if available
\urlstyle{same} % disable monospaced font for URLs
\hypersetup{
  pdftitle={Assignment 4},
  pdfauthor={Ankit Mithbavkar},
  colorlinks=true,
  linkcolor={blue},
  filecolor={Maroon},
  citecolor={Blue},
  urlcolor={Blue},
  pdfcreator={LaTeX via pandoc}}

\title{Assignment 4}
\author{Ankit Mithbavkar}
\date{2025-11-24}

\begin{document}
\maketitle
\ifdefined\Shaded\renewenvironment{Shaded}{\begin{tcolorbox}[enhanced, frame hidden, boxrule=0pt, interior hidden, borderline west={3pt}{0pt}{shadecolor}, breakable, sharp corners]}{\end{tcolorbox}}\fi

\renewcommand*\contentsname{Table of contents}
{
\hypersetup{linkcolor=}
\setcounter{tocdepth}{3}
\tableofcontents
}
\hypertarget{about-the-data}{%
\section{About the Data}\label{about-the-data}}

barroso2021 is a dataset created by Barroso for his paper ``A
meta-analysis of the relation between math anxiety and math
achievement.'' The study compiles the results from 332 studies,
including 747 total effect sizes, in order to understand the relation
between math anxiety and math achievement on a multitude of data.

\hypertarget{load-the-data}{%
\section{Load the Data}\label{load-the-data}}

\begin{verbatim}
Warning: package 'tidyverse' was built under R version 4.3.3
\end{verbatim}

\begin{verbatim}
Warning: package 'tibble' was built under R version 4.3.3
\end{verbatim}

\begin{verbatim}
Warning: package 'tidyr' was built under R version 4.3.3
\end{verbatim}

\begin{verbatim}
Warning: package 'readr' was built under R version 4.3.3
\end{verbatim}

\begin{verbatim}
Warning: package 'purrr' was built under R version 4.3.3
\end{verbatim}

\begin{verbatim}
Warning: package 'dplyr' was built under R version 4.3.3
\end{verbatim}

\begin{verbatim}
Warning: package 'stringr' was built under R version 4.3.3
\end{verbatim}

\begin{verbatim}
Warning: package 'forcats' was built under R version 4.3.3
\end{verbatim}

\begin{verbatim}
Warning: package 'lubridate' was built under R version 4.3.3
\end{verbatim}

\begin{verbatim}
-- Attaching core tidyverse packages ------------------------ tidyverse 2.0.0 --
v dplyr     1.1.4     v readr     2.1.5
v forcats   1.0.0     v stringr   1.5.1
v ggplot2   4.0.0     v tibble    3.2.1
v lubridate 1.9.4     v tidyr     1.3.1
v purrr     1.0.4     
-- Conflicts ------------------------------------------ tidyverse_conflicts() --
x dplyr::filter() masks stats::filter()
x dplyr::lag()    masks stats::lag()
i Use the conflicted package (<http://conflicted.r-lib.org/>) to force all conflicts to become errors
\end{verbatim}

\begin{verbatim}
Warning: package 'psymetadata' was built under R version 4.3.3
\end{verbatim}

\begin{verbatim}
  es_id study_id                  author pub_year continent grade low_ability
1     1        6                 Hoffman     2010         1     5           2
2     2        6                 Hoffman     2010         1     5           2
3     3       19             Farquharson     2004         1     5           2
4     4       31 Beasley, Long, & Natali     2001         1     3           2
5     5       37         Novak & Tassell     2015         1     5           2
6     6       37         Novak & Tassell     2015         1     5           2
  teachers  ni          yi          vi
1        1  70 -0.40005965 0.014925373
2        1  70 -0.44769202 0.014925373
3        2  89  0.01000033 0.011627907
4        2 278 -0.28768207 0.003636364
5        1  30 -0.46604720 0.037037037
6        1  30 -0.69314718 0.037037037
\end{verbatim}

\hypertarget{clean-the-data}{%
\section{Clean the Data}\label{clean-the-data}}

Before working with the data, it's important to remove missing
continents and map the continents and grades to labels that can be
easily understood.

\begin{verbatim}
  es_id study_id                  author pub_year continent grade low_ability
1     1        6                 Hoffman     2010         1     5           2
2     2        6                 Hoffman     2010         1     5           2
3     3       19             Farquharson     2004         1     5           2
4     4       31 Beasley, Long, & Natali     2001         1     3           2
5     5       37         Novak & Tassell     2015         1     5           2
6     6       37         Novak & Tassell     2015         1     5           2
  teachers  ni          yi          vi continent_label    grade_label
1        1  70 -0.40005965 0.014925373      N. America Post-Secondary
2        1  70 -0.44769202 0.014925373      N. America Post-Secondary
3        2  89  0.01000033 0.011627907      N. America Post-Secondary
4        2 278 -0.28768207 0.003636364      N. America  6th-8th Grade
5        1  30 -0.46604720 0.037037037      N. America Post-Secondary
6        1  30 -0.69314718 0.037037037      N. America Post-Secondary
\end{verbatim}

\hypertarget{research-question}{%
\section{Research Question}\label{research-question}}

Performance anxiety is a universal feeling among all ethnicity,
cultures, and ages. Often the result of a lack of confidence, the fear
of not meeting a standard or goal can lead one to fulfill that fear. The
same can be said about math. If one lacks confident in their math
abilities and has high anxiety regarding them, their math achievement
will reflect such.

As a result, I seek to explore whether this assumption holds true across
all grade levels and continents.

\hypertarget{visualize}{%
\section{Visualize}\label{visualize}}

\includegraphics{Assignment4_files/figure-pdf/plot-analysis-1.pdf}

\hypertarget{conclusion}{%
\section{Conclusion}\label{conclusion}}

As predicted, the relationship between math anxiety and achievement is
largely negative across all groups (shown by the data points falling
below the dotted zero line). This means that for a majority of people
across the globe, regardless of grade level, experiencing higher math
anxiety leads to lower achievement.

Specifically focusing on high school students, our dataset includes
\textbf{98} distinct effect sizes. The average effect size for this
group is \textbf{-0.34}, suggesting a moderate negative correlation
between anxiety and performance during the high school years.



\end{document}
